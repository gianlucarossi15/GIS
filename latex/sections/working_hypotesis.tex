\section{Working Hypotesis}
The following working hypotheses for the development of this project are based on a careful analysis of the requirements.
\subsection{Considerations}
The primary focus is to develop a robust and scalable road network system that perfectly meets the needs of the Padova Province. First of all, it is very important to follow the standards and applying best practices in roads management. Additionally, the system will be designed as an application compatible with commonly used browsers like Chrome and Firefox. There will be also functionalities for specific use cases that may require customization using desktop GIS tools.
\subsection{Hypothesis}
Implementing a customized GIS web application, integrated with a PostgreSQL DBMS, it is possible to create an efficient roads management system meeting the specified requirements. The system will enable the province to manage the following items: the road registry, the vertical signs and the inspections of the road conditions . To achieve this, the following key features are required:
\begin{itemize}
\item Import and export of data in GDF format and other standard formats (GML, shape) to facilitate integration with existing data sources.
\item Update the graph and attributes of the network, while maintaining the possibility of saving the history of changes. In this way, a complete view of the roads network evolution over the time will always be available.
\item A system of queries that combines spatial and alphanumeric filters to enable users to perform complex queries.
\item Export of the query results to save and utilize the obtained data for future analysis.
\item Visualization of the map and ortophoto in the background to allow the analysis of data within the system.
\end{itemize}
In addition to these fundamental features, the system will address specific requirements within the roads network domain. It will enable the province to:
\begin{itemize}
\item Management of road signs like knowing their locations and attributes, insert new ones or remove those eliminated by technicians.
\item Detection and collection of the road conditions made by a specialized team. In particular is necessary to detect: the presence of malformations on the road surface, damage to road signs and damage to the safety devices.
\item The data of the road registry, of the signs and of the safety devices can be consulted by the technical offices of the province.
\end{itemize}
\subsection{Analysis of Risks and Constraints}
Before the development of the project, an analysis of potential risks and constraints will be conduct to verify that some risks  may affect the project’s success. These could include technical challenges, data quality issues, resource limitations, and compliance requirements. 