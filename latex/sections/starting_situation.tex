\section{Starting Situation}
To really understand the project, it is important to look at the starting situation, including the technologies, existing data, market analysis and constraints to satisfy.
\subsection{Technology}
The IT department of the city of Padova is in charge of the management of the information system using commercial software and the PostgreSQL database management system (DBMS).\\
However, a Geographic Information System (GIS) application for the management of the road network of the province is not currently developed: our goal is to build a customized GIS solution to fulfill the province’s requirements.
\subsection{Existing data}
The Veneto region already has a topographic database which follows the national standard corresponding to the NC5 level, based on the ETRF2000 reference system. This database provides informations about the province's roads registry.\\
Additionally, an ortophoto map at a scale of 1:5000 is accesible through WMS non-transactional services.
\subsection{Market analysis}
There are currently no packaged solutions that meet all these requirements. Therefore there is a big opportunity to develop a solution that meets perfectly the city's needs. 
The application would be appreciated by the citizens for  signaling road malfunctions with the goal to improve the status and efficiency of the overall road network.
\subsection{Constraints}
There are no particular time, monetary and business constraints.